\documentclass[../PiTest.tex]{subfiles}

\begin{document}

\section{Introduction}

	\subsection{Purpose of the document}
	This document describes the project, the analysis of the problem, its design, the choices taken, its use and how to extend its functionality.
	
	\subsection{Purpose of the product}
	This product has the purpose of simplify the use of security research software on a mobile context, allowing the user to access a complete set of tools from an easy to use interface on a mobile phone.\\
	The final product is going to offer the following functionality:
	\begin{itemize}
		\item Secure login through a REST API;
		\item Execute common tasks, as network and WiFi scanning;
		\item Execute predefined scripts;
		\item Execute arbitrary commands, give by the user;
		\item Connect to the device through a WiFi hot-spot;
		\item Control the device using an Android App client;
		\item Show the results of the commands to the user in a simple GUI.
	\end{itemize}
	
	\subsection{Useful references}
		\begin{itemize}
			\item\textbf{Project presentation}: \url{goo.gl/CrLc5Y};
			\item\textbf{GitLab documentation}: \url{docs.gitlab.com};
			\item\textbf{Git documentation}: \url{git-scm.org/documentation};
			\item\textbf{Android documentation}: \url{developers.android.com};
			\item\textbf{NodeJS documentation}: \url{nodejs.org/docs};
			\item\textbf{Docker documentation}: \url{docs.socker.com};
			\item\textbf{Arch Linux Wiki}: \url{wiki.archlinux.org};
			\item\textbf{Black Arch documentation}: \url{blackarch.org/guide};
			\item\textbf{Raspberry Pi documentation}: \url{raspberrypi.org/documentation};
			\item\textbf{Aircrack-ng documentation}: \url{aircrack-ng.org/documentation};
			\item\textbf{Wireshark documentation}: \url{wireshark.org/docs}.
		\end{itemize}
			
		\subsection{Repository}
		The repository of the project is a Git repository, hosted on GitLab at the following address: \url{gitlab.com/bertof/PiTest}.

\end{document}